\chapter[Organic Lake]{Heterotrophic resourcefulness and unusual sulphur biogeochemistry in a hypersaline Antarctic Lake}
\label{ch:org}
\acresetall

%---------------------------------------------------------------------------------------------
\section*{Co-authorship Statement}
A version of this chapter has been submitted as:\\

\textbf{Sheree Yau}, Timothy J. Williams, Federico M. Lauro,  Matthew Z. DeMaere, Mark V. Brown, John Rich, 
John A.E. Gibson, Ricardo Cavicchioli. 
Heterotrophic resourcefulness and unusual sulfur biogeochemistry in a hypersaline lake.
\emph{\underline{ISME Journal}}
submitted, 2013.

Contributions to this manuscript by other researchers is as follows.
Research was designed by Federico Lauro, Mark Brown, John Gibson and Ricardo Cavicchioli.
Sample collection was performed by Federico Lauro and Ricardo Cavicchioli.
Metagenomic sequence filtering, global assembly and annotation was performed by Matthew DeMaere.
Assistance in analysis of functional gene markers provided by Timothy Williams.

Apart from these contributions, I performed all other data analyses and interpretations.
\newpage

%----------------------------------------------------------------------------------------------

\section{Abstract}
Organic Lake is a shallow, marine-derived hypersaline lake in the Vestfold Hills, Antarctica that has the highest reported concentration of dimethylsulphide \ac{DMS} in a natural body of water.
To determine the composition and functional potential of the microbial community and learn about the unusual sulfur chemistry in Organic Lake, shotgun metagenomics was performed on size fractionated samples collected along a depth profile.
Eucaryal phytoflagellates were the main photosynthetic organisms.
Bacteria were dominated by the globally distributed heterotrophic taxa \emph{Marinobacter}, \emph{Roseovarius} and \emph{Psychroflexus}.
Candidate division RF3 was overrepresented at the oxycline and OD1 in the lake bottom.
The dominance of heterotrophic degradation coupled with low fixation potential indicates possible net carbon loss.
However, abundant marker genes for aerobic anoxygenic phototrophy, sulfur oxidation, rhodopsins and CO oxidation were also linked to the dominant heterotrophic bacteria and may indicate use of photo- and lithoheterotrophy as mechanisms for conserving organic carbon.
Similarly, a high genetic potential for the recycling of nitrogen compounds likely functions to retain fixed nitrogen in the lake.
\ac{DMSP} lyase genes (\emph{dddD}, \emph{dddL} and \emph{dddP}) were abundant indicating \ac{DMSP} is a significant carbon and energy source.
Unlike marine environments, \ac{DMSP} demethylases (\emph{dmdA}) were less abundant than \ac{DMSP} lyases indicating that \ac{DMSP} cleavage is the likely source of the high \ac{DMS} concentration.
Strategies of nutrient resourcefulness such as \ac{DMSP} cleavage, carbon mixotrophy (photoheterotrophy and lithoheterotrophy) and nitrogen remineralization in dominant Organic Lake bacteria are potentially important adaptations to nutrient constraints.
In particular, carbon mixotrophy reduces the extent of carbon oxidation for energy production allowing more carbon to be used for biosynthetic processes.
The study sheds light on how microbial communities and the functional processes they perform evolve in response to unusual environmental conditions.

\newpage

%---------------------------------------------------------------------------------------------
\acresetall
\section{Introduction}
Molecular biology approaches have proven useful for describing the diversity and gene content of microorganisms in Antarctic lakes and for inferring the functional roles of the taxa present \cite{Laybourn-Parry2007}.
However to date, only a few large scale shotgun metagenome studies have been performed on the Antarctic continent and in the surrounding Southern Ocean \cite{Wilkins2012a}. 
In the Vestfold Hills, metagenomics and metaproteomics have been used to study Ace Lake (68$^{\circ}$28$'$23.2$''$S, 78$^{\circ}$11$'$20.8$''$E) and Organic Lake (68$^{\circ}$27$'$23.4$''$S, 78$^{\circ}$11$'$22.6$''$E) \cite{Ng2010a, Lauro2011}.
For Ace Lake, a comprehensive assessment of the community structure, biogeochemical fluxes and responses to resource limitation have been described \cite{Lauro2011}.
The metabolism of abundant green sulphur bacteria \cite{Ng2010a} was found to play a central role in nutrient cycling and a mathematical model was developed that showed its dominance was dependent on synchronicity with the polar light cycle leading to absence of phage predation \cite{Lauro2011}.
For Organic Lake, a member of the virophage virus family was discovered that potentially regulates microbial loop dynamics \cite{Yau2011}. 
The Organic Lake virophage likely depends on phycodnaviruses (algal viruses) and it was predicted that the virophage would reduce infective phycodnaviruses leading to an increased frequency of algal blooms and thus carbon flux \cite{Yau2011}.
Virophage sequences were also identified in a range of aquatic metagenomes revealing that they are likely to play ecologically important roles in many aquatic systems \cite{Yau2011}. 
These studies on Ace and Organic lakes both used shotgun metagenomics and illustrate the value of adopting a metagenomics approach for learning about microbial ecology in Antarctic environments.

Due to the polar light cycle, phototrophic growth in Antarctic environments is relatively high in summer and negligible in winter \cite{Laybourn-Parry2005} and requires microbial life to survive under long periods under a scarcity of resources.
To overcome this limitation, Eucaryotic phytoflagellates in Ace Lake engage in carbon mixotrophy by grazing on bacterioplankton to supplement their carbon requirements in the winter \cite{Laybourn-Parry2005}.
Heterotrophic marine bacteria can be similarly resourceful by exploiting light energy through photoheterotrophic processes including \ac{AAnP} or via use of rhodopsins, or lithoheterotrophy such as oxidation of carbon monoxide \cite{Moran2007b}.

Organic Lake is shallow (6.8 m) and has variable surface water temperatures ($-$14 to $+$15 $^{circ}$C) while remaining sub-zero throughout most of its depth \cite{Franzman1987b, Gibson1991, Roberts1993, Gibson1999}.
The salt and marine biota in the lake originate from seawater that was trapped in a basin ~3 000 y BP \cite{Zwartz1988, Bird1991}. 
The bottom waters of Organic Lake are unusual due to the absence of hydrogen sulphide and the high concentration of the volatile gas \ac{DMS} \cite{Deprez1986, Franzmann,1987b, Gibson1991, Roberts1993a, Roberts1993b}. 
Concentrations of \ac{DMS} as high as 5 000 nM have been recorded in Organic Lake \cite{Gibson1991}, 100 times the maximum concentration recorded from seawater in the adjacent Prydz Bay and at least 1 000 times that of the open Southern Ocean \cite{Curran1998}.
More than forty years ago atmospheric DMS was proposed to have a regulatory effect on global cloud cover as it is a precursor of cloud condensation nuclei (Lovelock and Maggs, 1972; Charlson et al., 1987).

%------------------------------------------------------------------------------------------

\section{Materials and methods}

%-----------------------------------------------------------------------------------------


\section{Results and discussion}

%------------------------------------------------------------------------------------------------------

\section{Acknowledgements}
