\begin{landscape}
\begingroup
\footnotesize
\begin{longtable}{p{2.5cm}p{3.5cm}p{2cm}p{3cm}p{2.5cm}p{5.5cm}p{3.5cm}}
\caption[PCR-based studies of Antarctic Lakes]{Studies of Antarctic Lakes that have made use of PCR amplification and sequencing of marker genes. The list presented here has attempted to be comprehensive but some studies some may have been inadvertantly missed. DV, McMurdo Dry Valleys; VH, Vestfold Hills; Syo, Syowa Oasis; SI, Signy Island; AP, Antarctic Peninsula; KGI, King George Island; LI, Livingston Island; CFB, \emph{Cytophaga}, \emph{Flavobacteria}, \emph{Bacteroidetes} group; \emph{Alpha}-, \emph{Alphaproteobacteria}; \emph{Beta}-, \emph{Betaproteobacteria}; \emph{Gamma}-, \emph{Gammaproteobacteria}; \emph{Delta}-, \emph{Deltaproteobacteria}; \emph{Epsilon}-, \emph{Epsilonbacteria}; \emph{Actino}-, \emph{Actinobacteria}; \emph{Cyano}-, \emph{Cyanobacteria}; SRB, sulphate-reducing bacteria; FISH, fluorescence \emph{in situ} hybridisation; FAME, fatty-acid methyl ester.
}
\label{tab:pcr_lakes}
\\
\toprule
\textbf{Site} & \textbf{Environment} & \textbf{Techniques} & \textbf{Organisms} & \textbf{Key processes} & \textbf{Notes} & \textbf{Reference}\\
\midrule
\endfirsthead
\multicolumn{7}{c}
{\tablename\ \thetable\ -- \textit{Continued from previous page}} \\
\toprule
\textbf{Site} & \textbf{Environment} & \textbf{Techniques} & \textbf{Organisms} & \textbf{Key processes} & \textbf{Notes} & \textbf{Reference}\\
\midrule
\endhead
\bottomrule \multicolumn{7}{r} {\textit{Continued on next page}} \\
\endfoot
\bottomrule
\endlastfoot
Lakes Bonney, 
Hoare, Fryxell, Joyce, Miers, Vanda, DV & Fresh to hypersaline, permanently ice-covered & 16S, \emph{amoA} libraries & \emph{Beta}-, \emph{Gamma}- & Ammonia  & Nitrifying bacterial \emph{amoA} detected in all lakes. In meromictic lakes, the population of \emph{Beta}– and  \emph{Gamma}- vertically stratified. Majority of nitrifying bacteria were \emph{Beta}-.& \cite{Voytek1999}  \\

Lake Bonney, DV & Hypersaline, meromictic, permanently ice-covered, separated into east and west lobes & \emph{nifH} library of ice aggregate material and microbial mats, nitrogenase assay & \emph{Cyano}-, \emph{Gamma}-, \emph{Alpha}-, \emph{Delta}- & N$_2$ fixation & Nitrogenase activity low compared to temperate environments. Heterotrophs responsible for 10--30\% of nitrogenase activity. Heterotrophs likely microaerophilic. & \cite{Olson1998}  \\

Lake Bonney, cyanobacterial mats, DV & Hypersaline, meromictic, permanently ice-covered, separated into east and west lobes & 16S library of ice sediments, hybridisation of probes & \emph{Cyano}-, \emph{Acidobacterium}/\emph{Holophaga}, green non-sulphur bacteria & Phototrophy and heterotrophy & Probes designed from 16S clone library of bacteria in the sediment in the ice matched that of the surrounding mats. & \cite{Gordon2000} \\

Lake Bonney, DV & Hypersaline, meromictic, permanently ice-covered, separated into east and west lobes & 18S libraries of watercolumn & \emph{Cryptophyta}, \emph{Chlorophyta}, \emph{Stramenopiles}, \emph{Haptophyta}, \emph{Choanoflagellida}, \emph{Alveolata}, fungi, ciliates & Photosynthesis & Population vertically stratified. Crytophytes dominant in the shallow water and haptophytes in the mid-depths and chlorophtes in the deeper waters. Stramenopiles replaced haptophytes during polar night. & \cite{Bielewicz2011} \\

Ekho, Organic, Deep Lakes, VH & Hypersaline. \textbf{Ekho and Organic}: meromictic and ice-covered $\sim$9 months of the year. \textbf{Deep}: holomictic, never freezes. & 16S libraries of sediment & \textbf{Organic}: \emph{Cyano}-/chloroplasts, CFB, \emph{Gamma}-, \emph{Alpha}-, \emph{Halobacteriales}, \emph{Actino}-. \textbf{Ekho}: \emph{Cyano}-/chloroplasts, CFB, \emph{Firmicutes}, \emph{Alpha}-, \emph{Gamma}-, \emph{Verrucomicrobiales}, \emph{Spirochaetales}. \textbf{Deep}: \emph{Halobacteriales}, \emph{Gamma}- & Heterotrophy & No phylotypes common to all samples. Distribution of bacterial classes similar between Ekho and Organic with \emph{Roseovarius} common to both. \emph{Marinobacter} and \emph{Halomonas} common to Organic and Deep. & \cite{Bowman2000b} \\

Lake Vida, DV & Hypersaline, meromictic, permanently ice-covered & 16S, 18S \acs{DGGE}, 16S library of ice core & \textbf{16S}: \emph{Actino}-, CFB, \emph{Gamma}-, \emph{Cyano}-, OD1, TM7, \emph{Firmicutes}, \emph{Planctomycetales}. \textbf{18S}: \emph{Chlorophyta}, fungi, \emph{Bacillariophyta}, \emph{Apicomplexa}, \emph{Cercozoa}, \emph{Chrysophyceae}, \emph{Ciliophora} & Phototrophy, heterotrophy & Cell density highest at the surface. Phylogeny shows \emph{Marinobacter} related to Lake Bonney isolate and bacterial sequences are similar to marine and polar organisms. & \cite{Mosier2007}  \\

Suribati-Ike, Syo & Hypersaline, meromictic, sulphidic anoxic bottom waters. & 16S libraries of halocline & \emph{Marinobacter}, \emph{Halomonas}, \emph{Pseudomonas}, \emph{Halocella} & Heterotrophy & \emph{Marinobacter} isolates capable of DMSO-respiration were relatives of those detected in lake water. Bacteria from the water column unable to respire nitrate. & \cite{Matsuzaki2006} \\

Clear, Pendant, Scale, Ace, Burton Lakes, Taynaya Bay, VH & Saline, meromictic lakes, high levels of sulphides (120--$>$250 mmol kg$^{-1}$ & 16S libraries of anoxic sediment & \textbf{Bacteria}: \emph{Firmicutes}, \emph{Cyano}-/chloroplasts, CFB, \emph{Delta}-, \emph{Alpha}-, \emph{Planctomycetes}, \emph{Gamma}-, green non-sulphur bacteria, \emph{Chlamydiales}, \emph{Verrucomicrobia}, \emph{Actino}-. \textbf{Eucarya}: 2.5\% of clones. \textbf{Archaea}: \emph{Methanosarcina barkerii}, unknown \emph{Euryarchaeota} group equidistant from \emph{Thermoplasma}, \emph{Methanomicrobiales}, \emph{Halobacteriales} & Sulphate reduction, methanogenesis, phototrophy, aerobic heterotrophy & Lakes with similar physico-chemical and limnological traits had more similar microbial communities. & \cite{Bowman2000a}  \\

Lake Fryxell, DV & Brackish, meromictic, permanently ice-covered & \emph{pufM} libraries, \acs{DGGE}, RT-PCR of \emph{pufM} transcripts & \emph{Alpha}-, \emph{Beta}- related to purple non-sulphur bacteria and aerobic anoxygenic phototrophs & Anoxygenic photosynthesis & Vertical stratification of the community down the water column. Purple and green sulphur bacteria not detected despite the high sulphide. \emph{pufM} transcripts only found below 9 m even though \emph{pufM} genes found throughout the water column. &  \cite{Karr2003} \\

Lake Fryxell, DV & Brackish, meromictic, permanently ice-covered & 16S \acs{DGGE} of water column & \emph{Methanoculleus}, \emph{Methanosarcina}, unclassified \emph{Euryarchaeota}, \emph{Methanosarcinales}-group and marine benthic group C-like \emph{Crenarchaeota} & Hydrogenotrophic methanogenesis, potential anoxic methanotrophy & Diverse population of methanogenic \emph{Euryarchaeota}, unclassified \emph{Euryarchaeota} and divergent \emph{Crenarchaeota} detected in sediments and water column. & \cite{Karr2006} \\

Nurume-Ike, Syo & Saline, meromictic & 16S library of anoxic sediment & \textbf{Archaea}: marine benthic group and unclassified \emph{Euryarchaeota}. \textbf{Bacteria}: \emph{Alpha}-, \emph{Delta}-, \emph{Planctomycetes}, \emph{Cyano}-/chloroplast, \emph{Gamma}-, \emph{Actino}-, CFB, \emph{Verrucomicrobia} and \emph{Spirochaetes} & Heterotrophy & Distribution of bacterial classes similar to lake sediment in VH except \emph{Alpha}- relatively overrepresented and \emph{Firmicutes} underrepresented. & \cite{Kurosawa2010} \\

Heywood Lake, Shallow Bay, SI & \textbf{Heywood}: ice covered for $\sim$9 months of the year, eutrophic due to organic inputs from seals, separated into two basins by shallow inlet. \textbf{Shallow}: coastal marine, ice covered during winter & Archaeal 16S, universal 16S libraries of anoxic sediment. Northern blots with methanogenic archaeal probes & \textbf{Heywood blots}: \emph{Methanomicrobiales}, \emph{Methanogenium}, \emph{Methanosarcinales}, \emph{Methanosaeta}. \textbf{Shallow blots}: \emph{Methanosarcinales}, \emph{Methanomicrobiales}, \emph{Methanococcoides}. \textbf{Heywood Archaea}: \emph{Methanosaeta}, \emph{Methanogenium}. \textbf{Shallow Archaea}: \emph{Methanogenium}, \emph{Methanolobus}, \emph{Methanococcoides}. \textbf{Heywood SRB}: \emph{Desulfovibrio}, \emph{Desulfotalea}/\emph{Desulforhopalus}, \emph{Desulfobulbus}, \emph{Desulfobacteriaceae}. \textbf{Shallow Bay SRB}: \emph{Desulfotalea}/\emph{Desulforhopalus}, \emph{Desulfobacterium}, \emph{Desulfobulbus}, \emph{Desulfobacteriaceae} & Acetoclastic and hydrogenotrophic methanogenesis, sulphur and metal oxidation, sulphate reduction & Methanogeneis and sulphate reduction detected at both sites. Diversity of methanogenic \emph{Archaea} extremely low. Methanogenic archaea 34\% and 0.2\% of community in Heywood Lake and Shallow Bay respectively. SRB 0.9\% and 14.7\% of community in Heywood Lake and Shallow Bay respectively. & \cite{Purdy2003}  \\

Sombre Lake, SI & Freshwater, ice-covered for $\sim$9 months of the year, oligotrophic, N and P limited & 16S libraries, DGGE, 16S libraries, FAME analysis of isolates, FISH of water column profile & \textbf{16S of isolates}: \emph{Beta}-, \emph{Firmicutes}, \emph{Actinobacteria}, \emph{Alpha}-, \emph{Gamma}-. \textbf{FAME}: \emph{Firmicutes}, \emph{Actino}-, \emph{Gamma}-, \emph{Beta}-, \emph{Alpha}-. \textbf{Clones}: \emph{Actino}-, CFB, \emph{Beta}-, \emph{Alpha}-, \emph{Spirochaetales}, \emph{Delta}-, \emph{Gamma}-, \emph{Verrucomicrobia}. \textbf{FISH}: \emph{Beta}-, CFB, \emph{Alpha}-, \emph{Gamma}-. \textbf{DGGE}: \emph{Actinobacteria}, CFB, \emph{Beta}- & Heterotrophic mainly respiratory metabolism & Relative abundances shown by clone libraries and FISH the same. Few genera common to culture-dependent and independent techniques. 16S isolate library and 16S clone library were significantly different. 16S clone library covers the largest spread of phyla but is missing \emph{Firmicutes}. Overall \emph{Beta}-. & \cite{Pearce2003a}  \\

Moss Lake, SI & Freshwater, ice-covered for $\sim$9 months of the year, oligotrophic, N and P limited & 16S DGGE and FISH of water column & \emph{Beta}-, CFB, \emph{Alpha}-, \emph{Gamma}-, \emph{Actino}-, \emph{Cyano}-. $<$1\% of cells hybridised with archaeal FISH probe & Heterotrophy, mainly respiratory metabolisms & Very little vertical stratification of population. 16S sequences similar to temperate and cold aquatic systems. & \cite{Pearce2003b} \\

Moss, Sombre, Heywood Lakes, SI & Freshwater, ice-covered for $\sim$9 months of the year, oligotrophic to eutrophic status  & 16S \acs{DGGE} of water column profile over the winter to summer transition & Not determined & Not determined & Lakes were physically and chemically stratified in winter, mixed in summer. Variation in bacterial community structure correlated with lake chemistry. Bacterial community still unstable during holomixis. & \cite{Pearce2005a} \\

Heywood Lake, SI & Freshwater, ice-covered for $\sim$9 months of the year, eutrophic due to organic inputs from seals, separated into two basins by shallow inlet. & 16S libraries, \acs{DGGE}, 16S libraries, FAME analysis of isolates, FISH of water column profile  & \textbf{16S clones}: \emph{Beta}-, \emph{Alpha}-, \emph{Actino}-. \textbf{FAME}: \emph{Actino}-, \emph{Firmicutes}, \emph{Gamma}-, \emph{Alpha}-. \textbf{FISH}: \emph{Beta}-, CFB, \emph{Gamma}-, \emph{Alpha}-. \textbf{\acs{DGGE}}: \emph{Actino}-, CFB, Gram-positives, \emph{Beta}-. & Heterotrophy, mainly respiratory metabolisms, phototrophy & Clone library coverage 71.7\%. Similar genera to Moss and Sombre Lakes. \emph{Actino}- and marine \emph{Alpha}- enriched compared to oligotrophic lakes while \emph{Cyano}- underrepresented. Species eveness is higher than Sombre or Moss Lakes. & \cite{Pearce2005b} \\

Lakes Boeckella, Esperanza, Flora, Encantado, Chico, Ping\"{u}i, Hope Bay, AP. Lakes L, M, W, Z, KGI & Freshwater, oligotrophic except Ping\"{u}i and Boeckella which were eutrophic and mesotrophic respectively & 18S \acs{DGGE} of surface water (20--3 \textmu{}m) & \emph{Chrysophyta}, \emph{Chlorophyta}, \emph{Dictyochophyceae}, \emph{Bacillariophyceae}, \emph{Cerozoa} & Photosynthesis & Molecular surveys showed a greater level of diversity exists than can be determined by light microscopy. Lake communities varied depending on trophic status. Lakes in both regions shared bands belonging to \emph{Chrysophyta} although they were 220 km apart. \emph{Dictyochophyceae} and \emph{Cercozoa} restricted to oligotrophic lakes. & \cite{Unrein2005} \\

Lakes Boeckella, Esperanza, Flora, Encantado, Chico, Ping\"{u}i, Hope Bay, AP. Lakes W and Z, KGI. & Freshwater, oligotrophic except Ping\"{u}i and Boeckella which were eutrophic and mesotrophic respectively. & 16S \acs{DGGE} of surface water (20--3 \textmu{}m) & CFB, \emph{Actino}-, \emph{Beta}-, \emph{Cyano}- & Heterotrophy, photosynthesis & Cluster analysis showed Lake communities from the Hope Bay formed one group while Lakes Chico, Ping\"{u}i and Boeckella formed another subgroup with KGI lakes. 63.7\% of variance is explained by axis 1 and 2 of Canonical Correspondence Analysis (40.4\% phosphate, dissolved inorganic nitrogen and pH; 23.3\% dissolved inorganic nitrogen). Temporal variation is not as pronounced as differences due to trophic status. & \cite{Schiaffino2009}  \\

Lakes Limnopolar, Midge, Chester, Chica, Turbio, Somero, Refugio, LI & Fresh to saline, all oligotrophic except for Refugio, which was eutrophic & 16S \acs{DGGE} of from surface water & CFB, \emph{Alpha}- & Heterotrophy, phototrophy & Cluster analysis showed deep lakes of the plateau grouped together while Somero and Refugio were separate groups. Over 90\% of variance was explained by chemical parameters related to trophic status and salinity. &  \cite{Villaescusa2010} \\

Lake Vostok & Largest subglacial lake, isolated from surface 420,000 years & 16S library of accretion ice core from 3,590 m & \emph{Alpha}-, \emph{Beta}-, \emph{Actinomycetes} & Potential heterotrophy & No \emph{Archaea} were amplified with archaeal primers. No biological incorporation of selected substrates & \cite{Priscu1999}  \\

Lake Vostok & Largest subglacial lake, isolated from surface 420,000 years & 16S library of accretion ice core from 3,590 and 3,603 m and isolation of bacteria & \emph{Alpha}-, \emph{Beta}-, \emph{Firmicutes}, \emph{Actino}-, CFB & Potential heterotrophy & Bacteria appear related to isolates from similarly cold environments. No \emph{Archaea} were amplified using archaeal primers. & \cite{Christner2001}  \\

Lake Vostok & Largest subglacial lake, isolated from surface 420,000 years & 16S, \emph{cbbL}/\emph{rbcL} and \emph{hoxV-hupL} library of accretion ice core from 3,561 m & \emph{Hydrogenophilus themoluteolus} & Potential hydrogenotrophy & Thermophilic chemolithoautotrohpic \emph{Hydrogenophilus thermoluteolus} 16S rRNA, \acs{RuBisCO} and NiFe-hydrogenase genes detected. &  \cite{Lavire2006} \\

Lake Vostok & Largest subglacial lake, isolated from surface 420,000 years & 16S library of Vostok drilling fluid recovered from 4 depths of the bore hole & \emph{Sphingomonas}, potential contaminants related to human/animal pathogens or saprophytes and environmental contaminants & Hydrocarbon degrading heterotrophs & New contaminant bacteria identified that were associated with hydrocarbon-based drilling fluid. &  \cite{Alekhina2007} \\

\end{longtable}
\endgroup
\end{landscape}
