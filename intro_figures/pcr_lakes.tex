\begin{landscape}
\begingroup
\footnotesize
\begin{longtable}{p{3cm}p{3cm}p{2cm}p{3cm}p{3cm}p{6cm}p{2cm}}
\caption[PCR-based studies of Antarctic Lakes]{Studies of Antarctic Lakes that have made use of PCR amplification and sequencing of marker genes.
}
\label{tab:pcr_lakes}
\\
\toprule
\textbf{Site} & \textbf{Environment} & \textbf{Techniques} & \textbf{Organisms} & \textbf{Key processes} & \textbf{Notes} & \textbf{Reference}\\
\midrule
\endfirsthead
\multicolumn{7}{c}
{\tablename\ \thetable\ -- \textit{Continued from previous page}} \\
\toprule
\textbf{Site} & \textbf{Environment} & \textbf{Techniques} & \textbf{Organisms} & \textbf{Key processes} & \textbf{Notes} & \textbf{Reference}\\
\midrule
\endhead
\bottomrule \multicolumn{7}{r} {\textit{Continued on next page}} \\
\endfoot
\bottomrule
\endlastfoot
Lakes Bonney, Hoare, Fryxell, Joyce, Miers and Vanda, McMurdo Dry Valleys & Fresh to hypersaline, permanently ice-covered & 16S and \emph{amoA} libraries & \emph{Betaproteobacteria}, \emph{Gammaproteobacteria} & Ammonia oxidation & Nitrifying bacterial \emph{amoA} detected in all lakes.
In meromictic lakes, the population of \emph{Beta}– and  \emph{Gammaproteobacteria} vertically stratified. 
Majority of nitrifying bacteria were \emph{Betaproteobacteria}.
 & \cite{Voytek1999}  \\
 &  &  &  &  &  &   \\
 &  &  &  &  &  &   \\
 &  &  &  &  &  &   \\
\end{longtable}
\endgroup
\end{landscape}
