\chapter[General Introduction]{General Introduction: Molecular Microbial Ecology of Antarctic Lakes}
\label{ch:intro}

%Coauthorship-----------------------------------------------------------------------------------

\section*{Co-Authorship Statement}
\addcontentsline{toc}{section}{Co-authorship Statement}

Sections of this chapter has been published as:\\

David Wilkins, \textbf{Sheree Yau}, Timothy Williams, Michelle Allen, Mark V. Brown, Matthew Z. DeMaere,
Federico M. Lauro and Ricardo Cavicchioli. (2012)
Key Microbial Drivers in Antarctic Aquatic Environments.
\textit{\underline{FEMS Microbiology Reviews}} 
(in press).\\

I contributed the section of the publication entitled \emph{Antarctic Lakes} excluding the subsection, \emph{Microbial mats as microcosms of Antarctic life}.
Most of this material is contained in section~\ref{in:oases} \emph{Antarctic Coastal Oases and Glacial Lakes}, section~\ref{in:mol} \emph{Molecular Approaches Used in Antarctic Lake Systems}, section~\ref{in:insights} \emph{Molecular Insights into Antarctic lakes} and section~\ref{in:pcrlimits} \emph{Limitations of Taxonomic Surveys} of this chapter.
\newpage

%-----------------------------------------------------------------------------------------------

\section{Introduction}
Antarctica is a ``frozen desert'' of constant low temperature, little precipitation and long polar light cycles where only specially adapted organisms can survive.
Only 0.4\% of the total ice area of Antarctica (12.3 $\times$ 10$^6$ km$^2$) is seasonally ice-free, comprising exposed mountains and coastal areas. %ref
Life is concentrated on the few ice-free coastal oases where liquid water is present in hundreds of lakes and ponds.
Lake biota is microbially dominated with few or no metazoans present \cite{Laybourne-Parry1997}.
The reduced complexity means it is possible to encompass a large proportion of the diversity present through the use of molecular-based techniques.
The lakes inhabit a continuum of environmental factors, presenting themselves as ``natural laboratories'' where comparisons can be made between lakes that vary in a property of interest. 
Meromictic lakes similarly provide an opportunity to describe microbial populations along chemical gradients, but within a single water body. 
This makes Antarctic lakes ideal model ecosystems to examine microbial influence on geochemistry by relating taxa to particular processes\cite{Laybourne-Parry2007}.

This introduction will describe the Antarctic lakes, their microbiology and review the molecular-based microbiological research that has been conducted on the lakes.
As this thesis focused on two lakes in the Vestfold Hills, more emphasis will be given to describing research from this study site.

%---------------------------------------------------------------------------------------------

\section{Antarctic Coastal Oases and Glacial Lakes}
\label{in:oases}
In Antarctica, perenially available liquid water is confined to coastal lakes, subglacial and epiglacial lakes.
Subglacial and epiglacial lakes appear to be prevalent with at least 145 subglacial lakes identified \cite{Siegert2005}.%check (Gibson, 2006; Cavicchioli, 2007; Pearce, 2009)
These include the largest, Lake Vostok. %name them.
Epiglacial lakes are. %describe them.

Coastal oases, where lakes are found exposed on the rocky land, include the Vestfold Hills, Bunger Hills, Larsemann Hills, Syowa Oasis, Schirmacher Oasis, Grearson Hills and McMurdo Dry Valleys
 in East Antarctica, the West Antarctic Peninsula and the sub-antarctic islands.
As Ad\'{e}lie penguins create nests from rocks, these few coastal oases are the only locations on the continent where they are able to breed.

Antarctic lakes span a wide range of physical and chemical properties from freshwater to hypersaline, ice-covered to perenially melted and permanently stratified (meromictic) to mixed (holomictic).
Most lakes are largely isolated due to long periods of ice-cover, and may be truly closed systems if ice-cover is permanent.
The age of water varies considerably, with the subglacial outflow from Blood Falls estimated to be 1.5 million years old \cite{Mickucki2009}, the water in Ace Lake about 5000 years old
 \cite{Rankin1999}, and Lake Miers water less than 300 years old \cite{Green1988}. %consider removing lake Miers. 
Organisms inhabiting the lakes may be examples of ancient life or undiscovered endemic species.

Of these locations, the best studied lake systems are those of the McMurdo Dry Valleys, The Vestfold Hills and the subantarctic islands.
%refer to table or map of the region
%How long have people done research on these lakes?

%--------------------------------------------------------------------------------------------

\section{The Vestfold Hills, East Antarctica}
The Vestfold Hills is a rocky ice-free region of approximately 400 km$^2$ on the eastern shore of the Prydz Bay, East Antarctica in the Australian Antarctic Territory (fig:vestfold map) \cite{Gibson1999}.
The region was first sighted and named in 1935\cite{Law1959}.
Only intermittent expeditions occurred in the area until the establishment of Davis Station (68$^{\circ}$33$'$S, 78$^{\circ}$15$'$E) in 1957 \cite {Law1959}. 
There Vestfold Hills region was immediately noted for its extensize ice-free land and the numerous lakes.\cite{Johnstone1973}.

The Australian Antarctic Data Centre lists more than 3000 water bodies mapped in the Vestfold Hills, ranging in area from 1 to 8,757,944 m\^{}2.%check this fact.
More than 300 lakes and ponds have been described, including approximately 20\% of the world's meromictic lakes \cite{Gibson, 1999}. %check this fact.
Fjords connected to the ocean also cut across the Vestfold Hills. 
Some of these are large, such as Ellis Fjord which is 10 km long, up to 100 m deep and has become a stratified system due to its restricted opening to the ocean \cite{Burke1988}.
The region was formed approximately 10,000 years ago when the retreat of the continental ice-shelf lead to isostatic uplift of the land \cite{Burton1981}. 

The lakes origated from water trapped in the exposed rocky depressions.
Many of the coastal lakes are marine-derived, having been separated from the marine environment 3,000--7,000 years ago \cite{Gibson1999} and are predominantly saline or hypersaline \cite{Burke1988}.
The latter are formed due to concentrated by ablation (evaporation and sublimnation). %ref 
Freshwater lakes near the continental ice shelf were likely already above sea-level as the ice receded and are not marine-derived \cite{Laybourne-Parry1992} \cite{Bronge1996}.
Lakes closer to the coast, such as Rookery Lake, may still occasionally experience marine inputs although most have completely separated from the sea. %ref

All lakes may receive water inputs from precipitation, from the ice-shelf and glacial melt streams \cite{Burton1981}. 
This can cause freshwater to seasonaly overlay some saline lakes as the ice-cover thaws.
Glacial meltwater flushing has lead some lakes that were originally marine-derived, such as Clear Lake, to become fresh \cite{Pickard1986}\cite{Bird1991}.
%Show a map of some of the main lakes in the Vestfold Hills
%Nutrient status? (see Burton 1981) temperature? weather? major ions? (see Burton1981). recorders of climate change in the sediment and in the location of the chemocline,what occurs in each zone
Since their formation each lake has followed a separate physical evolution depending on the local geography and now have very different chemical and physical properties. 

\subsection{Biology of the Vestfold Hills}

\subsection{Lakes of the Vestfold Hills}

Much early work was dedicated to the biology of the Vestfold Hills.
What was interesting and special?
What was the picture they had a microbial life?

\subsection{History of studies in Organic Lake}
\subsection{History of studies in Ace Lake}

%------------------------------------------------------------------------------------------

\section{Cultivation-based Antarctic microbiology}

Early microbiological surveys began X.
Bacteria were detected by cultivation or by microscopy. 
Identification was limited to those species that could be isolated and appropriate identification tests performed.
Eucarya were identified with microscopy based approaches.


\subsection{Eucarya}
\subsection{Bacteria}
\subsection{Archaea}
\subsection{Viruses}

%------------------------------------------------------------------------------------------

\section{Molecular Approaches Used in Antarctic Lake Systems}
\label{in:mol}
The majority of molecular-based studies of Antarctic aquatic microbial communities have made use of PCR amplification of small subunit ribosomal RNA sequences to survey the diversity of Bacteria
 and in some cases Archaea and Eucarya. %table
Microbial composition has been determined by cloning and sequencing of rRNA gene amplicons exclusively 
\cite{Bowman2000a, Bowman2000, Gordon2000, Christner2001, Purdy2003, Karr2006, Matsuzaki2006, Kurosawa2010,Bielewicz2011}, 
although most studies have also made use of denaturing gradient gel electrophoresis (DGGE) to provide a molecular ``fingerprint'' of the community 
\cite{Pearce2003, Pearce2003, Karr2005, Pearce2005, Pearce2005, Unrein2005, Glatz2006, Mikucki2007, Mosier2007, Schiaffino2009, Villaescusa2010}.
Functional genes have also been targeted using PCR amplification to assess the potential of biochemical processes occurring, such as nitrogen fixation \cite{Olsen1998}, 
ammonia oxidation \cite{Voytek1999}, anoxygenic photosynthesis \cite{Karr2003}, and dissimilatory sulfite reduction \cite{Karr2005, Mikucki2009}. %add in new studies

%-----------------------------------------------------------------------------------------

\section{Insights from Antarctic Molecular Studies}
\label{in:insights}
\subsection{Bacterial Diversity: Adaptation to Unique Physical and Chemical Conditions}
The vast majority of molecular studies of Antarctic lakes have focused on bacteria.
Consistent with the wide range of physical and chemical properties of Antarctic lakes, a large variation in species assemblages have been found.
While exchange of microorganisms must be able to occur between lakes that are in close vicinity to each other, 
the picture that has emerged from the data to date is that microbial populations are relatively unique to each type of isolated system. 
Nonetheless, certain trends in bacterial composition are also apparent.

Focusing on the similarities, lakes of equivalent salinities tend to have similar communities.
Hypersaline lakes from the Vestfold Hills \cite{Bowman2000b} and McMurdo Dry Valleys \cite{Glatz2006, Mosier2007} were all dominated by Gammaproteobacteria and members of the Bacteroidetes
 as well as harboring lower abundance populations of Alphaproteobacteria, Actinobacteria, and Firmicutes.
The surface waters of saline lakes resemble marine communities dominated by Bacteroidetes, Alphaproteobacteria and Gammaproteobacteria,
 but divisions such as Actinobacteria and specific clades of Cyanobacteria have been found to be overrepresented compared to the ocean \cite{Lauro2011}.
Sediments from saline lakes in the Vestfold Hills \cite{Bowman2000a} and Nuramake-Ike in the Syowa Oasis \cite{Kurasawa2010} were very similar, 
containing in addition to the surface clades, Deltaproteobacteria, Planctomycetes, Spirochaetes, Chloroflexi (green non-sulfur bacteria), Verrucomicrobia and representatives of candidate divisions.
Plankton from freshwater lakes were characterized by an abundance of Betaproteobacteria, 
although Actinobacteria, Bacteroidetes, Alphaproteobacteria and Cyanobacteria were also prominent \cite{Pearce2003, Pearce2005, Pearce2005, Schiaffino2009}. 

\subsubsection{Bacterial Diversity Defined by Nutrients}
Differences in bacterial community structure are also influenced by nutrient availability.
In studies of freshwater lakes in the Antarctic Peninsula and the South Shetland Islands, cluster analysis of DGGE profiles grouped together lakes of similar trophic status 
\cite{Schiaffino2009, Villaescusa2010}.
Most of the variance in community structure could be explained by related chemical parameters such as phosphate and dissolved inorganic nitrogen.
Similarly, three freshwater lakes, Moss, Sombre and Heywood on Signy Island are alike except that Heywood Lake is enriched by organic inputs from seals.

Bacterial composition in each lake changed from winter to summer and this was again correlated to variation in physico-chemical properties \cite{Pearce2005}. 
The bacterial population of Heywood Lake had shifted from a dominance of Cyanobacteria towards a greater abundance of Actinobacteria and marine Alphaproteobacteria \cite{Pearce2005}.
This hints at a link between a copiotrophic lifestyle in the Heywood Lake Actinobacteria and inhibition of Antarctic freshwater Cyanobacteria by eutrophication. 
This type of study exemplifies how inferences can be made about taxa and function by examining population changes over time and over gradients of environmental parameters.

\subsubsection{Bacterial Biogeography}
The relative isolation and diverse chemistries of the lakes facilitates biogeographical and biogeochemical studies. 
The anoxic and sulfidic bottom waters of some meromictic lakes form due to a density gradient that precludes mixing. 
Although sedimentation from the upper aerobic waters may occur, 
there is little opportunity for interchange of species with the bottom water of lakes allowing for greater divergence in community composition as nutrients can become depleted 
and products of metabolism can accumulate.
As a result, distinct distributions of bacterial groups can inhabit these strata, and different types of microorganisms can be found in equivalent strata in different lakes. 
A good example of this is the presence of common types of purple sulfur bacteria (Chromatiales)and green sulfur bacteria (Chlorobi) 
in some meromictic lakes and stratified fjords in the Vestfold Hills \cite{Burke1988},
compared to diverse purple non-sulfur bacteria in Lake Fryxell in Victoria Land \cite{Karr2003}. 
%Check if these aren't Roseobacters
In Lake Bonney, the east and west lobes harbor overlapping but distinct communities in the suboxic waters \cite{Glatz2006}.
The east lobe was dominated by Gammaproteobacteria and the west lobe by Bacteroidetes, illustrating how divergent communities can form from the same seed population. 
In contrast, ice communities are more readily dispersed by wind, aerosols and melt-water. 
16S rRNA gene probes designed from bacteria trapped in the permanent ice-cover of Lake Bonney hybridized to microbial mat libraries sourced up to 15 km away \cite{Gordon2000}.
This demonstrates how a single lake may encompass microorganisms that are geographically dispersed, while also harboring others that have restricted niches and are under stronger selection pressure.


\subsubsection{Bacterial Diversity of Lake Vostok}
Subglacial systems, such as Lake Vostok, have been isolated from the open environment for hundreds of thousands to millions of years \cite{Siegert2001}.
As a result they provide a reservoir of microorganisms that may have undergone significant evolutionary divergence from the same seed populations that were not isolated by the Antarctic ice cover. 
The uniqueness of these types of systems also creates a conundrum for studying them. 
Lake Vostok is approximately 4 km below the continental ice-sheet making it extremely difficult to determine suitable means for accessing the lake without inadvertently contaminating it with biological
 or chemical matter \cite{Inman 2005, Wingham2006, Lukin2011, Gramling2012, Jones2012}. 
To date, molecular microbial studies have concentrated on the accretion ice above the ice-water interface \cite{Priscu1999, Christner2000}.
Accretion ice has been found to contain a low density of bacterial cells from Alphaproteobacteria, Betaproteobacteria, Actinobacteria and Bacteroidetes divisions closely allied to other cold environments.
Molecular signatures of a thermophilic Hydrogenophilu sspecies were also identified in accretion ice 
raising the possibility that chemoautotrophic thermophiles were delivered to the accretion ice from hydrothermal areas in the lake’s bedrock \cite{Bulat2004, Lavire2007}.
However, interpretation of results from samples sourced from the Lake Vostok bore hole are very challenging as it is difficult to differentiate contaminants from native Vostok microorganisms.
From a study that assessed possible contaminants present in hydrocarbon-based drilling fluid retrieved from the Vostok ice core bore hole, 
six phylotypes were designated as new contaminants \cite{Alekhina2007}. 
Two of these were Sphingomonas phylotypes essentially identical to those found in the accretion ice-core \cite{Christner2000},
 which raises question about whether bacterial signatures identified from the ice-cores are representative of Lake Vostok water,
 and further highlights the ongoing problem of causing forward contamination into the lake.

\subsection{Archaea: Methanogens and Haloarchaea}
Archaea have been detected mainly in anoxic sediments and bottom waters from lakes that range in salinity from fresh to hypersaline, 
and those with known isolates are affiliated with methanogens or haloarchaea \cite{Bowman2000a, Bowman2000b, Purdy2003, Kurasawa2010, Lauro2011}.
%relate how this is different to marine deep waters where Crenarchaea are abundant.
Anoxia allows for the growth of methanogenic archaea that mineralize fermentation products such as acetate, and H$_2$ and CO$_2$ into methane, thereby performing an important step in carbon cycling.
The acetoclastic methanogens thrive in environments where alternative terminal electron acceptors such as sulfate and nitrate have been depleted. 
%This may be why there are none in Organic Lake as there is still sulfate.

One example of this is Lake Heywood where methanogenic archaea were found to comprise 34\% of the total microbial population in the freshwater sediment, 
the majority of which were Methanosarcinales which include acetate and C1-compound utilizing methanogens \cite{Purdy2003}. 
Both H$_2$:CO$_2$ (Methanogenium frigidum) and methylamine/methanol (\textit{Methanococcoides burtonii}) utilizing methanogens were isolated from Ace Lake 
\cite{Franzmann1992, Franmann1997} providing opportunities for genomic analyses \cite{Saunders2003, Allen2009} and a host of studies addressing molecular mechanisms of cold adaptation 
(e.g. \cite{Cavicchioli2006, Williams2011}.

In general, archaeal populations appear to be adapted to their specific lake environment.
Sediments from saline lakes of the Vestfold Hills were inhabited by members of the Euryarchaeota typically found in sediment and marine environments 
with the phylotypes differing between the lakes examined \cite{Bowman2000a}. 
While a phylotype similar to Methanosarcina was identified, the majority were highly divergent. 
Similarly, Methanosarcina and Methanoculleus were detected in Lake Fryxell but other members of the Euryarchaeota and Crenarchaeota (a single sequence) were divergent, 
clustering only with marine clones \cite{Karr2006}. 
Based on the lake chemical gradients and the location of these novel phylotypes in the water column 
the authors speculated these archaea may be have alternative metabolisms such as anoxic methanotrophy or sulfur-utilization. 

In sediments from Lake Nurume-Ike in the Langhovde region, 205 archaeal clones grouped into three phylotypes, 
with the predominant archaeal clone being related to a clone from Burton Lake in the Vestfold Hills, while the other two did not match to any cultivated species \cite{Kurasawa2010}. 
Consistent with these observations, from a metagenomic study that involved the analysis of 9 million genes, a high level of divergence was found for the archaea present in the bottom waters of Ace Lake; 
the majority of which did not match to known methanogens including \textit{M. frigidum} and \textit{M. burtonii} that were isolated from the lake \cite{Lauro2011}. 
However, high levels of methane are present in Ace Lake bottom waters and this is likely to have been produced gradually by the methanogenic community, 
and retained in the lake due to the very low potentialfor aerobic methane oxidation and the apparent absence of anaerobic methane oxidizing (ANME)Euryarchaeota \cite{Lauro2011}.

In hypersaline lakes where bottom waters do not become completely anoxic, methanogens are not present and archaea have extremely low abundance. 
For example, only two archeael clones of the same phylotype were recovered from deep water samples from Lake Bonney \cite{Glatz2006}, 
and Organic Lake in the Vestfold Hills had an extremely low abundance of archaeal clones related to Halobacteriales \cite{Bowman2000b}. 
In contrast to these stratified hypersaline lakes, the microbial community in the extremely hypersaline Deep Lake isdominated by haloarchaea \cite{Bowman2000b}. 
Many of the clones identified from Deep Lake are similar to \textit{Halorubrum} (formerly \textit{Halobacterium}) \textit{lacusprofundi} which was isolated from the lake \cite{Franzmann1988}. 

\subsection{Eucarya Perform Multiple Ecosystem Roles}

Single-celled Eucarya are important members of Antarctic aquatic microbial communities.
In many Antarctic systems, eucaryal algae are the main photosynthetic organisms and in others, only heterotrophic protists occupy the top trophic level. 
As eucaryal cells are generally large with characteristic morphologies, microscopic identifications have been used. 
However, microscopy is unable to classify smaller cells such as nanoflagellates with high resolution, although these may constitute a high proportion of algal biomass.
For example, five morphotypes of Chrysophyceae, evident in Antarcticlakes were unidentifiable by light microscopy but were able to be classified using DGGE and DNA sequencing \cite{Unrein2005}.
Consistent with this, molecular studies specifically targeting eucaryal diversity \cite{Unrein2005, Mosier2007, Bielewicz2011} have identified a much higher level of diversity than previously suspected,
 and the studies have discovered lineages not previously known to be present such as silicoflagellates \cite{Unrein2005} and fungi \cite{Mosier2007, Bielewicz2011}.

Most eucarya in Antarctic lakes are photosynthetic microalgae that are present in marine environments with a wide distribution including chlorophytes, haptophytes, cryptophytes and bacillariophytes.
Molecular methods have afforded deeper insight into the phylogenetic diversity within these broader divisions and have revealed some patterns in their distribution. 
Using 18S rRNA gene amplification and DGGE, the same chrysophyte phylotypes were identified in lakes from the Antarctic Peninsula and King George Island 
despite being 220 km apart \cite{Unrein2005} indicating these species may be well-adapted to Antarctica or highly dispersed.
Similarly, an unknown stramenopile sequence was detected throughout the 18S rRNA clone libraries of Lake Bonney 
demonstrating a previously unrecognized taxon occupied the entire photic zone in the lake \cite{Bielewicz2011}. 
In constrast, other groups showed distinct vertical and temporal distributions with cryptophytes dominating the surface, 
haptophytes the midwaters and chlorophytes the deeper layers during the summer while stramenopiles increased in the winter \cite{Bielewicz2011}. 
Further studies are necessary to determine the basis for apparent specific adaptations of some species to particular lakes or lake strata, and for the cosmopolitan distribution of others.
Here, molecular based research of the kind that has been applied to bacteria such as functional gene surveys will undoubtedly help answer these questions.
%add in new studies of photosynthetic genes

\subsection{functional}

\subsection{whole ecosystem}
%----------------------------------------------------------------------------------------------------------------------


\section{Limitations of Taxonomic Surveys}
\label{in:pcrlimits}

In terms of determining taxonomic profile of environments, detection of SSU sequences has hugely expanded what is known about microbes present in natural systems.

Inferring functional potential from taxonomic surveys can be problematic due to species or strain level differences in otherwise related bacteria.
For example, the majority of the Gammaproteobacteria in hypersaline lakes were relatives of \textit{Marinobacter} suggesting that this genus is particularly adapted to hypersaline systems
\cite{Bowman2000b, Glatz2006, Matsuzaki2006, Mosier2007}.
Nonetheless, \textit{Marinobacter} species from different lakes appeared biochemically distinct
 as isolates from hypersaline lake Suribati-Ike were all able to respire dimethylsulfoxide (DMSO) but not nitrate \cite{Matsuzaki2006}. 
In contrast, those from the west lobe of Lake Bonney were all able to respire nitrate \cite{Ward1997}. 
Interestingly, in the east lobe of the same lake, nitrate respiration was inhibited although a near-identical \textit{Marinobacter} phylotype was present; 
it was speculated that the inhibition may have been caused by an as yet unidentified chemical factor \cite{Ward2005, Glatz2006}. 

This also applies to Eucarya, as the influence of flagellates on ecosystem function is not necessarily clear-cut as they can simultaneously inhabit several trophic levels. 
For instance, in Ace Lake the mixotrophic phytoflagellate \textit{Pyramimonas gelidocola} derives a proportion of its carbon intake through bacterivory \cite{Bell2003} 
but in the nearby Highway Lake, it uptakes dissolved organic carbon \cite{Laybourn-Parry2005}. 
This again illustrates potential limitations for deriving ecosystem level functions from taxonomic studies alone, even with taxa that appear physiologically straightforward. 

%--------------------------------------------------------------------------------------------------------------------

\section{`-omics' Approaches}
Metagenomic studies have assessed both the taxonomic composition and genetic potential of lake communities, and in some cases have linked function to specific members of the community 
\cite{Lopez-Bueno2009, Ng2010, Lauro2011, Yau2011, Varin2012}.%how to fit your work in??
When coupled with functional ``omic'' techniques (to date metaproteomics has been applied, but not metatranscriptomics or stable isotope probing), 
information has also been gained about the genetic complement that has been expressed by the resident populations \cite{Ng2010 Lauro2011, Yau2011}.
\subsection{Viruses}

%-----------------------------------------------------------------------------------------------------------------

\section{Objectives}
Overall, this study aimed to use metagenomic and metaproteomic approaches to gain an integrative understanding of the Ace and Organic Lake ecosystems. 
Using this methodology, not only can the taxonomic composition of the lakes be determined but also the functional potential of the microbial population and insight into the active members of the community.
The objectives of the research were:

\begin{enumerate}
\item
  To determine the microbial and viral composition of the lake
  communities.

\item
  To determine the functional potential of the lake biota.

\item
  To reconstruct as much genomic information as possible of dominant taxa and to infer their physiology and ecological role.

\item
  To integrate environmental and biological data and model the lake microbial interactions and geochemical processes.

\end{enumerate}
