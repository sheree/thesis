\section{Introduction}

\subsection{Why Study Antarctic Aquatic Systems? Research Rationale}

The Antarctic Continent is a ``cold desert'' defined by constant
low temperatures and low precipitation.

In this harsh environment, only specially adapted organisms can
survive.

Those that are able to inhabit the continent are concentrated on
the few ice-free coastal oases such as the McMurdo Dry Valleys, the
Larseman Hills, the Bunger Hills, the Vestfold Hills and the
Antarctic Peninsula where liquid water is present in a myriad of
lakes and ponds.

These lakes have a wide range of physical and chemical properties
from freshwater to hypersaline, ice-covered to permanently liquid,
stratified to mixed and also ages.

\subsection{The Vestfold Hills, East Antarctica}

\subsection{History of Organic Lake Research}

\subsection{History of Ace Lake Research}

\subsection{Insights From Molecular-based Studies of Antarctic Aquatic Terrestrial Systems}

\subsection{Limitations of Previous Studies}

