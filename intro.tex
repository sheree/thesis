were able to be phylogenetically classified \cite{Coolen2004a, Coolen2004b, Coolen2006, Coolen2008}.

The contemporary microbiota of Ace Lake and Organic Lake have also been the subject of numerous studies.
A thorough review was published on Ace Lake's limnology by \citet{Rankin1999} that includes descriptions of its biology.
Within this review, twenty-one publications were cited that examined the microbial community through use of microscopic analysis, flow cytometry, pigment and lipid chemistry, biochemical rate measurements and cultivation \cite{Rankin1999}.
%Benthic communities consist of microbial mats (Hand1981, Wright1981, Dartnall1997) including algae (Dartnall1997) and diatoms \cite{Roberts1996} with associated ciliates, copepods and rotifers (Dartnall1997).
%The only zooplankton known to inhabit Ace Lake itself is the copepod \emph{Paralabidocera antarctica}.
%Algae microscopy: diatoms \cite{Roberts1996}, Dartnall1997? Hand1981? Wright1981, Volkman1988, Mancuso1990, Laybourn-Parry and Periss1995, Peris1995.
%Pigments and lipids: Volkman1988, Mancuso1990
%Other ciliates: Laybourn-Parry1997
%Flow cytometry used on cyanobacteria: Rankin1997, Rankin1998
%Microscopy of bacteria: Burton and Barker1979, Burch1988., Burke and Burton1998b
%Lipid analysis: Volkman1986,1988, Mancuso1990
%Heterotrophic bacteria: Hand and Burton1981, Rankin1998
%Isolates: \emph{Methylosphaera hansonii}?, Bowman1997b; \emph{Carnobacterium} Franzmann1991a; cell wall-less spirochaete, Franzmann and Rohde 1992, Franzmann and Dobson 1992; anaerobic coiled bacterium, Franzmann and Rohde1991; M burtonii and frigidum, Franzmann1992,1997; Shewanella frigidum from mats Bowman1997a.
%Archaea found by lipid analysis and determined their density and activity of methanogens, same for SRB, Mancuso1990. 
%SRB outcompete methanogens higher in the water column, Franzmann1991a
%Primary production measured by: Wright and Burton 1981, Laybourn-Parry and Perris1995 comparable to other antarctic environments.
%methanogenesis production was measured and found to be at the limit of detection (Franzmann1991b)
%Methanotrophs detected (Bowman1997b).
%list: Burton and Barker 1979, Hand1981, Wright1981, Volkman1988, Burch1988, Mancuso1990, Franzmann1991a, Franzmann and Rohde1991, Franzmann and Dobson1992, Franzmann1992, Laybourn-Parry and Perriss1995, Periss1995, Roberts1996, Franzman1997, Bowman1997a, Bowman1997b, Dartnall1997, Rankin1997, Laybourn-Parry1997, Rankin1998, Burke and Burton1998b
Subsequently, several studies have examined the Ace Lake microbial community \emph{in situ} using non-molecular cultivation-independent approaches.
Three studies have focussed on viral ecology in Ace Lake using \ac{TEM} and fluorescence microscopy \cite{Laybourn-Parry2001, Madan2005, Laybourn-Parry2007}.
These studies have found features of the viral community that may be unique to polar lakes such as high rates of lysogeny during winter and spring (up to 71\% of cells) \cite{Laybourn-Parry2007}.
Two studies have measured flagellate grazing rates using fluorescently labelled microspheres or bacteria and noted high rates of bacteriovory in the phytoflagellate \emph{Pyramimonas gelidicola}, which may be a specific adaptation to low light availability \cite{Bell2003, Laybourn-Parry2005}.

There are fewer publications concerning Organic Lake's biota than for Ace Lake's.
The earliest publication was the first report of a choanoflagellate in a hypersaline environment \cite{vandenHoff1986}.
An in-depth limnological study of Organic Lake followed, which identified and enumerated \emph{Eucarya} and \emph{Bacteria} by microscopic examination as well as by isolation \cite{Franzmann1987b}.
The isolates from Organic Lake included \emph{Dunaliella} sp., \emph{Halomonas subglaciecola}, \emph{Halomonas meridiana}, \emph{Psychroflexus gondwanensis}, \emph{Salegentibacter salegens} and numerous unidentified heterotrophic bacteria \cite{Franzmann1987b}.
Continued work on these isolates are represented by eight publications \cite{Burch1983, Franzmann1987a, McMeekin1988b, James1990, Dobson1991, Dobson1993, Bowman1998, McCammon2000}.
Subsequent work on Organic Lake targeted towards the \emph{in situ} microbial community has included determination of grazing rates in choanoflagellates \cite{Marchant1993}, measurements of the seasonal abundances of \emph{Flavobacteria} and \emph{Halomonas} spp. using antibody probes \cite{James1994},  microscopy-based surveys of the phyto- and bacterioplankton \cite{Roberts1996, Perriss1997} and a geolipid survey of the sediments \cite{Rogerson1996}.

There have been few molecular analyses on Ace and Organic Lakes before the metagenomic and metaproteomic work presented in this thesis.
The sediments of both lakes have been surveyed for their \ac{SSU} diversity \cite{Bowman2000a, Bowman2000b} and molecular taxonomy has been performed on a cultivation-based survey of heterotrophic bacteria from microbial mats \cite{VanTrappen2002}.
For Ace Lake, sequencing of \ac{SSU} genes has been performed from both extant and fossil DNA of methanogenic \emph{Archaea}, methylotrophic bacteria and \ac{GSB} \cite{Coolen2004a, Coolen2004b, Coolen2006, Coolen2008}.
Heterotrophic bacteria containing antifreeze proteins have been isolated from Ace Lake and determined by 16S \ac{rRNA} gene sequencing to be related to \emph{Marinomonas protea}, \emph{Stenotrophomonas maltophilia}, \emph{Enterobacter agglomerans} and \emph{Pseudoalteromonas} sp. \cite{Gilbert2004}.
Cultivation-independent  molecular studies have revealed an unexpected level of diversity existed in the contemporary microbial community, particularly in the heterotrophic \emph{Bacteria} and \emph{Archaea}, which were previously largely undescribed \cite{Bowman2000a, Bowman2000b, Coolen2004a, Coolen2004b, Coolen2006}.
They also illustrates a lack of knowledge of the whole microbial community as no molecular studies prior to the metagenomic and metaproteomic programme that this thesis work was a part of \cite{Ng2010a, Lauro2011} have encompassed the entire water column.
Using high-thoughput molecular-based analyses enabled an unprecedented level of detail in the description to be obtained of the microbial community composition and function.
%The first studies of Ace Lake's palaeobiology used hydrocarbons as biomarkers of the past microbial communities and was able to identify large changes had occured over the lake's history \cite{Volkman1986, Volkman1988}.
%The microfossils of the rotifier \emph{Notholca} and dinoflagellate and chrysophyte cysts indicate early freshwater lake community was diverse and ecologically developed as it spanned multiple trophic levels \cite{Swadling2001}.
%Frustules of freshwater diatom species in sediment cores of Ace Lake correspond to its early freshwater phase and later shift to a saline marine-derived community is reflected in the presence of \emph{Navicula} and \emph{Chaetoceros} species in sediment cores (Roberts and Mcminn1999
%\citet{Schouten2001} analysed stable isotope carbon biomarkers in the Ace Lake sediments, focussing on biogeochemical processes inferred to occur in Ace Lake (methanogensis, phototrophic sulphide oxidation and sulphate reduction).
%Screening for 18S rRNA genes and lipid biomarkers from an Ace Lake sediment core was able to identify six unknown species of \emph{Prymnesiophyta} that were likely abundant in the lake over 10,000 BP \cite{Coolen2004a}.
%A similar study examining archaeal 16S rRNA genes found phylotypes in the younger sediments and were most closely related to environmental, predominantly euryarchaeotal clones, none of which were known methanogenic or methanotrophic lineages however lipid analyses detected methanogens in these depths \cite{Coolen2004b}.
%Interestingly, 3 phylotypes from the oldest sediments did include relatives of methanogenic \emph{Methanosarcinales} indicating methanogens were present in the early freshwater lake \cite{Coolen2004b}.
%Very few methanogens were detected in the water column by \acs{PCR} \cite{Coolen2004b}.
%Analysis of the same sediment core examining 16S rRNA gene diversity of the \ac{GSB} and carotenoids found the same \ac{GSB} phylotype extant in the anaerobic bottom waters is the same as in the sediments cores from $\sim$9,400 BP \cite{Coolen2006}.
%Viral morphotypes found in Ace Lake included tailed phages and large $\sim$300 nm icosahedra \cite{Laybourn-Parry2001}. 
%Measurements of seasonal virus and cell abundances showed \acp{VLP} correlate negatively with phototrophic flagellates and with light \cite{Madan2005} but correlate positively with primary production \cite{Laybourn-Parry2007}. 

%-----------------------------------------------------------------------------------------------------------------

\section{Objectives}
The overall aim of this thesis was to explore the microbial communities of Antarctic lakes from an ecosystem level perspective by combining metagenomics, metaproteomics and physico-chemical data.
Ace Lake and Organic Lake, two lakes in the Vestfold Hills, were chosen as the study sites as there are extensive historic environmental records available for these lakes.
Furthermore, as meromictic lakes, differences in the microbial population were able to be examined along the vertical gradients within the lakes.
Finally, as marine derived lakes, comparisons can be made between the lake communities and the Southern Ocean to determine specific adaptions to the lake environment.

The specific objectives of the research were:

\begin{enumerate}
\item 
  Develop epifluorescence microscopy and metaproteomic methods to complement metagenomic sequencing.

\item
  Determine the microbial and viral composition of Antarctic lake communities, their functional potential and infer the ecological roles of populations in the community.

\item
  Integrate environmental and biological data to model the lake microbial interactions and biogeochemical processes.

\end{enumerate}
