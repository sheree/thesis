\begin{abstract}
The Vestfold Hills is a coastal Antarctic oasis, a rare ice-free region on the continent containing hundreds of lakes including a high density of meromictic (permanently stratified) lakes. These lakes are ideal model ecosystems for studying microbial diversity, evolution and biogeochemical cycling as microbial populations can be examined along physico-chemical gradients in a single water body. 
Two marine-derived meromictic lakes in the Vestfold Hills, Ace Lake and Organic Lake, were chosen as study sites as extensive historic, physico-chemical and biological datasets exist for both.

To study their microbial ecology, analysis of genetic material randomly sequenced from the environment (metagenomics) was performed, which evaluates both microbial composition and metabolic potential.
To support metagenomic inferences, methods for analysis of proteins from environmental samples (metaproteomics) and microscopy were developed and applied to both lakes.
Metaproteomic analysis indicated active community members and processes while microscopy determined microbial/viral abundances and morphology.

An integrative approach combining metagenomic, metaproteomic and physico-chemical data enabled comprehensive descriptions of the lake ecosystems, including the identification of taxa not previously known to inhabit the lakes and determination of biogeochemical cycles.
A complete genome was reconstructed of a member of the newly described virophage viral family and near complete genomes of phycodnaviruses.
The virophage likely `preys' on phycodnaviruses that infect eucaryotic phytoflagellates.
A model of virophage--phycodnavirus--algae population dynamics predicted the presence of a virophage increases the frequency of algal blooms and thus overall nutrient release.
Virophage signatures were detected in other aquatic environments indicating they play a previously unrecognised role in other environments. 
In Organic Lake, genes involved in DMSP cleavage, photo- and lithoheterotrophy and nitrogen remineralisation were associated abundant heterotrophic bacteria indicating these processes are adaptations to overcome nutrient constraints in Organic Lake.
Specifically, employing photo- and lithoheterotrophy enables more carbon to be used for biosynthesis rather than energy generation thereby conserving carbon in the lake, while recycling of nitrogen limits its loss.
These molecular-based discoveries shed light on how previously unrecognised taxa and metabolic processes are involved in adaptation to the unique Antarctic lake environment.


\end{abstract}
\newpage
