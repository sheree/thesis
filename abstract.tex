\begin{abstract}
The Vestfold Hills is a coastal Antarctic oasis, a rare ice-free region containing a high density of meromictic (permanently stratified) lakes. These lakes are ideal model ecosystems as their microbial communities exist along physico-chemical gradients, allowing populations to be correlated with geochemical factors. 
As extensive historic, physico-chemical and biological datasets exist for Ace Lake and Organic Lake, two marine-derived meromictic lakes, they were chosen as study sites for molecular-based analysis of their microbial communities. 

Analysis of genetic material randomly sequenced from the environment (metagenomics) was performed to determine taxonomic composition and metabolic potential.
To support metagenomic inferences, methods were developed for performing microscopy on lake water samples and for the identification of proteins from the environment (metaproteomics).
Metaproteomic analysis indicated active community members and processes, while microbial/viral abundances and morphology were determined by microscopy.
An integrative approach combining metagenomic, metaproteomic and physico-chemical data enabled comprehensive descriptions of the lake ecosystems.
This included the identification of taxa not previously known to inhabit the lakes and determination of biogeochemical cycles.

A complete genome was reconstructed of a member of the newly described virophage viral family and near complete genomes of phycodnaviruses.
The virophage likely `preys' on phycodnaviruses that infect eucaryotic phytoflagellates.
A model of virophage--phycodnavirus--algae population dynamics predicted the presence of a virophage increases the frequency of algal blooms and thus overall nutrient release.
Virophage signatures were detected in other aquatic environments indicating they play a previously unrecognised role in other environments. 
In Organic Lake, genes associated with heterotrophic bacteria involved in \acs{DMSP} cleavage, photoheterotrophy, lithoheterotrophy and nitrogen remineralisation were abundant, indicating these processes are adaptations to nutrient constraints.
Photo- and lithoheterotrophy enables carbon to be used for biosynthesis rather than energy generation thereby conserving carbon in the lake, while recycling of nitrogen limits its loss.
\acs{DMSP} apppears to be significant carbon and energy source and also the origin of high \acs{DMS} concentrations in Organic Lake. 
These molecular-based discoveries shed light on the role of previously unrecognised taxa and metabolic processes in unique Antarctic lake environments.


\end{abstract}
\newpage
