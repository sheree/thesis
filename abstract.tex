\begin{abstract}
The Vestfold Hills is a coastal Antarctic oasis, a rare ice-free region on the continent containing hundreds of marine-derived lakes. These lakes are microbially-dominated systems constrained by extremes of cold, salinity and light availability. Most Antarctic lakes are ice-covered for the majority of the year making them largely closed systems. The physico-chemical gradients that exist within meromictic (permanently stratified) lakes makes it possible to relate microbial taxa to abiotic variables. These factors make Antarctic lakes ideal model ecosystems for studying microbial diversity, evolution and biogeochemical cycles. 

Sequencing of ribosomal genes from the environment has revolutionised microbial ecology by revealing the immense diversity of microbial life. However, random sequencing of genetic material from the environment (metagenomics) allows the determination not only of microbial composition, but also metabolic potential. Historic, physical, chemical and biological data available for two meromictic lakes in the Vestfold Hills, Ace Lake and Organic Lake, indicate each has unique microbial populations and biogeochemical properties. Metagenomic sequencing was applied to these two lakes to gain insight into their microbial ecology. Analytical methods to support metagenomic inferences were also developed and applied. 
These included identification and quantification of proteins from environmental samples (metaproteomics) and microscopy.
The former indicated active community members and biochemical processes while the latter determined microbial/viral abundances and morphology.

An integrative approach combining metagenomic, metaproteomic and physicochemical data enabled a comprehensive description of the lake ecosystems including taxa that were previously unknown to occur in the lakes and capacity for nutrient cycling.
Complete genomes of phycodnaviruses and a member of the newly described virophage viral family were reconstructed.
The virophage likely `preys' on phycodnaviruses to complete it replication cycle thus playing a role in regulating algal population dynamics and nutrient release via cell lysis.
Virophage signatures were found in abundance in Ace Lake and detected in other aquatic environments indicating they have a role in other environments.
Potential for carbon cycling in Organic Lake indicated lower rates of carbon fixation than respiration.
However, abundant genes involved in photoheterotrophy linked to dominant bacterial lineages suggested mixotrophy enables carbon to be conserved within the lake system.
Similarly, a potential for recycling of reduced nitrogen appears to limit nitrogen loss.

These molecular-based discoveries allowed the role of previously unrecognised taxa and metabolic processes to be modelled enabling implications to be drawn about Antarctic and other aquatic environments.

\end{abstract}
\newpage
