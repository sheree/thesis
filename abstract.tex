\chapter*{Abstract}
%\addcontentsline{toc}{chapter}{Abstract}
%note abstract must not be more than 350 words and shall indicate the problem investigated, the procedures followed, the general results obtained and the major conclusions reached, but shall NOT contain any illustrative matter.

The Vestfold Hills is a coastal Antarctic oasis, a rare ice-free region on the continent containing hundreds of marine-derived lakes are found.
These lakes are microbially-dominated systems constrained by extremes of cold, salinity and the annual polar light cycle.
Differing local geographic features has led each lake to develop unique chemistries tightly linked to the resident microbial populations
Most Antarctic lakes are ice-covered for a large proportion of the year and so are largely closed systems.
Their physical isolation and the extreme environmental constraints imposed by the polar environment makes them potential reservoirs of novel taxa.
Furthermore, many lakes are stratified so the physical and chemical gradients that exist within a single system makes it possible to relate microbial taxa to physico-chemical variables.
These factors make Antarctic lakes ideal model ecosystems in which to study microbial diversity, evolution and influence on geochemistry.

It is well-known that understanding microbial ecology through use of culture-based techniques gives a limited picture of natural populations as most microbes are not readily culturable by standard methods.
Application of molecular biology techniques to the study of microbes directly from the environment has revealed the enormous diversity of microbial life.
One powerful approach is random high-throughput sequencing of genetic material directly from the environment (metagenomics), which allows the determination not only of the microbial community composition and structure, but also their metabolic potential.
The studies described in this thesis use metagenomic sequencing of microbial communities in stratified of saline lakes in the Vestfold Hills.

In order to describe the role of microbes in the ecology of the whole lake environment, analyses complementary to metagenomic sequencing were also developed and applied.
These included identification and quantification of proteins extracted directly from the environment (metaproteomics) and microscopy for direct counts of microbial and viral abundances and morphological examination.
Taken in combination with both historic and contemporary physico-chemical data, molecular-based information allowed a description of the lake ecosystem and also resulted in totally new insights into mechanisms of adaption to the Antarctic environment.
From these genomic discoveries, hypotheses of the role previously unknown taxa and functional genes have on the environment were developed and modelled.
These discoveries not only have implications for Antarctic environments, but potentially have broader implications for other aquatic systems.





% Add in another 100 words. Elaborate more on the methods and the problems.Not of each lake, but of the general difficulty perhaps of understanding whole ecosystems. conclusions.
