\chapter*{Abstract}
%\addcontentsline{toc}{chapter}{Abstract}
%note abstract must not be more than 350 words and shall indicate the problem investigated, the procedures followed, the general results obtained and the major conclusions reached, but shall NOT contain any illustrative matter.
Microbes are abundant biological entities that drive geochemical processes on the planet.
However, understanding of microbial ecology through use of traditional culture-based techniques gave a very limited picture of natural populations.
Application of molecular biology techniques to the study of microbes directly from the environment has revealed the enormous diversity of microbial life.
In addition, random high-throughput sequencing of genetic material directly from the environment (metagenomics) allows the determination not only of the microbial community composition and structure, but also of their metabolic potential.

This study has utilised metagenomic sequencing on stratified saline lakes in the Vestfold Hills, East Antarctica.
These are unique microbially dominated systems constrained by extremes of cold, salinity and the annual polar light cycle.
Most Antarctic lakes are ice-covered for a large proportion of the year and so are largely closed systems.
Their physical isolation and the environmental constraints imposed by the polar environment makes them potential reservoirs for novel taxa to be discovered.
The physical and chemical gradients that exist within these stratified lakes systems makes it possible to relate microbial taxa to physico-chemical variables.
These factors make Antarctic lakes ideal model ecosystems in which to study microbial evolution and how microbes influence geochemistry.

In order to further our picture of the role of microbes in the ecology of the whole lake environment, analyses complementary to metagenomic sequencing were developed and applied.
These included identification and quantification of proteins extracted directly from the environment (metaproteomics) and microscopy for direct counts of microbial and viral abundances and morphological examination.
Taken in combination with both historic and contemporary physico-chemical data, molecular-based information allowed a description of the lake ecosystem and functional potential and also resulted in totally new insights into mechanisms of adaption to the Antarctic environment.

Genomic discoveries have led to the generation of new hypotheses of how microbial life has adapted to the Antarctic environment.
It also has lead to development of models of population and nutrient dynamics.
These pave the way for future experiments to test these predictions.
These discoveries not only have implications for Antarctic environments, but potentially have broader implications for other aquatic systems.





% Add in another 100 words. Elaborate more on the methods and the problems.Not of each lake, but of the general difficulty perhaps of understanding whole ecosystems. conclusions.
