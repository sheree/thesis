\chapter*{Abstract}
%\addcontentsline{toc}{chapter}{Abstract}
%note abstract must not be more than 350 words and shall indicate the problem investigated, the procedures followed, the general results obtained and the major conclusions reached, but shall NOT contain any illustrative matter.
Since the discovery that the vast majority of microbial life is unculturable and viruses are likely the most abundant biological entities on the planet, microbial ecology has sought to understand the diversity of microbial life, determine its relative abundance and their role in their environment.
Application of molecular biology techniques to the study of microbes in their natural environment has shown the enormous diversity of microbial life.
Sequencing of genetic material directly from the environment (metagenomics) allows the determination not only of the microbial community composition and structure, but also of their metabolic potential.
This study has utilised metagenomic sequencing on saline lakes in the Vestfold Hills, East Antarctica.
These are unique microbially dominated systems constrained by extreme cold and salinity, as well as the annual polar light cycle.
These factors make Antarctic lakes ideal model ecosystems in which to study microbial evolution and how microbes influence geochemistry.
This study sought to describe the microbial community and genetic potential of Antarctic Lake systems.
Integration of analyses complementary to metagenomics, such as metaproteomics and microscopy, in combination with both historic and contemporary physico-chemical data allowed a whole ecosystem level description of the lakes studied.
Application of these molecular-based analyses to the Antartic lake systems has resulted in totally new insights into the microbial community and functional potential of the lakes.
Furthermore, discovery of completely novel viruses and metabolic processes was achieved.
These discoveries not only have implications for Antarctic environments, but potentially have broader implications for other aquatic systems.



% Add in another 100 words. Elaborate more on the methods and the problems.Not of each lake, but of the general difficulty perhaps of understanding whole ecosystems. conclusions.
