\begin{abstract}
The Vestfold Hills is a coastal Antarctic oasis, a rare ice-free region on the continent containing hundreds of marine-derived lakes. These lakes are microbially-dominated systems constrained by extremes of cold, salinity and light availability. Most Antarctic lakes are ice-covered for the majority of the year and are thus largely closed systems that often become meromictic (permanently stratified). The physical and chemical gradients that exist within an isolated system makes it possible to relate microbial taxa to abiotic variables. These factors make Antarctic lakes ideal model ecosystems to study microbial diversity, evolution and influence on geochemistry. 

Sequencing of ribosomal genes from the environment has revolutionised microbial ecology by revealing the immense diversity of microbial life. However, this approach does not directly describe the physiology and ecological roles of members in a community. Random sequencing of genetic material from the environment (metagenomics) allows the determination not only of the microbial composition, but also its metabolic potential. Historic, physical, chemical and biological data available for two meromictic lakes in the Vestfold Hills, Ace Lake and Organic Lake, indicate each has unique microbial populations and biogeochemical properties. Metagenomic sequencing was applied to these two lakes to gain insight into their microbial ecology. Analytical methods to support metagenomic inferences were also developed and applied. These included identification and quantification of proteins from environmental samples (metaproteomics), which indicates active community members and biochemical processes; as well as microscopy, for determination of microbial/viral abundances and morphology. 

Analysis of these lake ecosystems yielded extensive genetic information from taxa previously unknown in the lakes, in particular, phycodnaviruses and a member of the newly described virophage viral family. The combination of metagenomics, metaproteomics and physico-chemical data  enabled the ecological roles of taxa in the lakes and potential mechanisms of adaptation to the Antarctic environment to be determined. These analyses revealed viral predation, higher propensity of nutrient recycling and strategies of carbon conservation in Antarctic lake ecosystems. These molecular-based discoveries allowed the role of previously unrecognised taxa and metabolic processes to be modelled enabling implications to be drawn about Antarctic and other aquatic environments.

\end{abstract}
\newpage
