\chapter{General discussion, conclusions and future work}
\label{ch:conc}
\acresetall

\section{Summary of the major achievements of this work}

%Development of microscopy and metaproteomics applied to subsequent studies.
%These methods are not perfect but they'll do.
%GSB were able to be described totally without culturing.
Proof of concept of the approach.
Totally new discoveries have made testable hypotheses.

\section{Future work}
\subsection{Ace Lake}
%Dynamics over the whole year.
%What happens to the lake during the winter? 
%Can we isolate them.
%Are they endemic or do they change?
%Can we

\subsection{Virophages}
Since the discovery of the \ac{OLV} \cite{Yau2011} described in chapter \ref{ch:olv}, two other members of the virophage family have been reported.
The first of these is the Mavirus (for Maverick virus) so named for its evolutionary relationship with the Maverick/Politon class of eucaryotic transposons \cite{Fischer2011}.
Like Sputnik, Mavirus appears to have an absolute requirement for a giant `helper' virus of the \ac{NCLDV} group to replicate.
In this case \ac{CroV} is the helper virus, which replicates in the cellular host \emph{Cafeteria roenbergensis}, a marine heterotrophic nanoflagellate. 
Also like Sputnik, addition of Mavirus during \ac{CroV} infection leads severe drops in production of \ac{CroV} \cite{Fischer2011}.
The other virophage reported was called Sputnik 2 (its genome is virtually identical to Sputnik) and like Sputnik was also found in association with strain of mimivirus, named lentille virus \cite{Desnues2012}.
Unlike Sputnik, it was found both as a free virion and integrated in the lentille virus genome \cite{Desnues2012}.
Integration is localised to a shared 352 bp region encoding a collagen-like repeat-containing protein in Sputnik (corresponding to Sputnik V6 gene), lentille virus and mamavirus, but not in mimivirus  \cite{Desnues2010}.

The availability of these new virophages in culture affords a new perspective on \ac{OLV} and virophages in general.
In first place, the inability for Mavirus to infect without CroV and it detrimental effect on CroV strengthens the assumption that OLV would similarly have this 'virophage' phenotype \ac{OLPV}.
This is in part because Mavirus and CroV and quite divergent from Sputnik and mimivirus respectively, yet retain the same traits \cite{Fischer2010, Fischer2011}, suggesting the 'virophage' phenotype is common to the whole lineage.
Secondly, having several genomes for comparative analysis points to a some mechanistic reasons for why \ac{OLV} would truly function as the other virophages.
As both mimivirus and CroV encode a great deal of their own replication and transcription machinery, including complete \textsc{DNA} and \textsc{RNA} polymerases, they do not localise to the nucleus during infection but generate the viral factory in the cytoplasm \cite{LaScola2008, Fischer2011}.
Sputnik and Mavirus replicate exclusively in the giant virus factory and make use of the helper virus replication and transcription systems \cite{LaScola2008, Fischer2011}.
Furthermore, Sputnik and Mavirus share the promoter and other transcriptional control sequences of their helper viruses \cite{Claverie2009, Fischer2011}.
They therefore are primarily parasitising the resources of the helper viruses, which is could be the cause of reduced helper virus production \cite{Claverie2009, Fischer2012}.
\ac{OLPV}-1 encodes all eight \textsc{RNA} polymerase subunits (see GenBank accession HQ704802) indicating it like all other members of the mimivirus lineage would replicate independently in the cytoplasm thereby making it likely \ac{OLV} would similarly utilise its helper viruses molecular machinery exclusively with comparable detrimental effects.

 
%Shows which genes are core, do they integrate, mode of infection.

However, these additional data also raise tantalising questions about the role of virophage physiology, evolution and ecological role.

%%%OLV physiology questions
%Need to isolate to determine if detrimental
%Determine which OLPV type it infects.
%Verify OLPV infects pyramimonas.
%Verify OLV reduces infective particles.

%%%OLV dynamics questions
%Make an exclusion experiment to show that the dynamics change with and without the OLV. 
%The single cell screening experiment.
%and track them over a season.

%%%OLV distribution and evolution questions
%virophages seem to attack only mimivirus relatives.
%How common are they in the environment?
%They seem like an ancient lineage, how widespread are they? Do they infect other NCLDV


Need future work on organic lake whole lake
%Need to show that AAnP genes are expressed with RT-PCR, proteomics
%Future experiments can look at fluorescence of bchlA
%Cloning of the dddD, dddL genes into appropriate host to show activity.
%Look for inputs into the lake.
%Is allochtonous inputs feeding the lake?
%Need to show the dynamics of the lake
%Need to look at the rest of viral community 

Perspective of Antarctic Lake research from wetlab to molecular age.

Assessment of the value of molecular antarctic research
%Molecular data can be continued to be mined! 
%Serves as a record for prosterity
%Metagenomics is not enough! There needs to be functional data, single cell data, biochemical data.

